\usepackage{listings}           % refer to 'minted vs. texments vs. verbments'
\usepackage{xcolor}
\lstnewenvironment{python}[1][]{
  \definecolor{commentsColor}{rgb}{0.497495, 0.497587, 0.497464}
  \definecolor{keywordsColor}{rgb}{0.000000, 0.000000, 0.635294}
  \definecolor{stringColor}{rgb}{0.558215, 0.000000, 0.135316}
  \lstset{
    float=h,
    backgroundcolor=\color{white},                          % choose the background color; you must add \usepackage{color} or \usepackage{xcolor}
    basicstyle=\footnotesize,                               % the size of the fonts that are used for the code
    breakatwhitespace=false,                                % sets if automatic breaks should only happen at whitespace
    breaklines=true,                                        % sets automatic line breaking
    captionpos=b,                                           % sets the caption-position to bottom
    commentstyle=\color{commentsColor}\textit,              % comment style
    deletekeywords={},                                      % if you want to delete keywords from the given language
    escapeinside={\%*}{*)},                                 % if you want to add LaTeX within your code
    extendedchars=true,                                     % lets you use non-ASCII characters; for 8-bits encodings only, does not work with UTF-8
    frame=tb,	                   	                        % adds a frame around the code
    keepspaces=true,                                        % keeps spaces in text, useful for keeping indentation of code (possibly needs columns=flexible)
    keywordstyle=\color{keywordsColor}\bfseries,            % keyword style
    language=Python,                                        % the language of the code (can be overrided per snippet)
    otherkeywords={},                                       % if you want to add more keywords to the set
    numbers=left,                                           % where to put the line-numbers; possible values are (none, left, right)
    numbersep=5pt,                                          % how far the line-numbers are from the code
    numberstyle=\tiny\color{commentsColor}\noncopynumber,   % the style that is used for the line-numbers
    rulecolor=\color{black},                                % if not set, the frame-color may be changed on line-breaks within not-black text (e.g. comments (green here))
    showspaces=false,                                       % show spaces everywhere adding particular underscores; it overrides 'showstringspaces'
    showstringspaces=false,                                 % underline spaces within strings only
    showtabs=false,                                         % show tabs within strings adding particular underscores
    stepnumber=1,                                           % the step between two line-numbers. If it's 1, each line will be numbered
    stringstyle=\color{stringColor},                        % string literal style
    tabsize=2,                                              % sets default tabsize to 2 spaces
    title=\lstname,                                         % show the filename of files included with \lstinputlisting; also try caption instead of title
    columns=fullflexible,                                   % Using fixed column width produces nice alignment. I think it's ugly, though -> fullflexible
    literate={-}{-}1
             {*}{*}1
             {\ }{{\copyablespace}}1 {\ \ }{{\copyablespaceTwo}}1,
    #1                                            % Optional arguments
    }  
}{}


%%%%% Workarounds %%%%%

% list of listings title
\renewcommand*{\lstlistlistingname}{List of listings}

% Make Python-Listing line numbers uncopyable
\usepackage[space=true]{accsupp}
\newcommand{\noncopynumber}[1]{%
    \BeginAccSupp{method=escape,ActualText={}}%
    #1%
    \EndAccSupp{}%
}

% Fix spaces in Python listing (really necessary?)
\newcommand{\copyablespace}{\BeginAccSupp{method=hex,unicode,ActualText=00A0}\ \EndAccSupp{}}
\newcommand{\copyablespaceTwo}{\BeginAccSupp{method=hex, unicode, ActualText=00A000A0}\ \ \EndAccSupp{}}
\makeatletter
  \def\lst@Literatekey#1\@nil@{\let\lst@ifxliterate\lst@if
  \expandafter\def\expandafter\lst@literate\expandafter{\lst@literate#1}}
\makeatother

%%%%% /Workarounds %%%%%