% !TEX TS-program = lualatex
% !TEX encoding = UTF-8 Unicode

\documentclass[11pt,a4paper]{scrartcl}
\usepackage{fontspec}

%% this is already inside preamble/color-graphics-tables.tex %%
\usepackage{array,tabularx}
\usepackage{booktabs}
%% this is already inside preamble/color-graphics-tables.tex %%

% Graphics, Tables, Colors
\input{linebreaks-in-tables.tex}

% Metadaten %
\title{Linebreaks in Tables Example}

\begin{document}
\makeatletter
{\LARGE \textbf{\@title}}
\makeatother
\vspace{2.5em}


\begin{table}[ht]\centering
    \begin{tabular}{ll}
        Claim       & Vor/Nachteil                                                                                                                                                      \\\midrule
        Startseite  & \makecell[l]{\textminus{} Sehr überladen\\\textminus{} Auch für erfahrene Nutzer unübersichtlich\\\textminus{} Nicht konfigurierbar\\\textminus{} Viel Scrollen}  \\\addlinespace
        Showcases   & \makecell[l]{+ Inspiration für erfahrene Nutzer\\\textminus{} Inhalte nicht konfigurierbar}                                                                       \\\addlinespace
        Suchleiste  & \makecell[l]{\textminus{} Suchleiste ist schwer zu finden\\\textminus{} Suchvorschläge werden leicht\\\textit{versehentlich} anvisiert}                           \\\addlinespace
        Suche       & \makecell[l]{\textminus{} Sehr viele unpassende Suchergebnisse\\\textminus{} Bei manchen Spielen überladen mit DLCs}                                              \\\addlinespace
        Durchstöbern& \makecell[l]{\textminus{} Tag-Ansichten haben ein anders Layout\\\textminus{} Tags haben keine ersichtliche Sortierung}                                           \\
    \end{tabular}
    \caption{Claims-Problemszenarien}\label{tab:claims-problemszenarien}
\end{table}

\end{document}






